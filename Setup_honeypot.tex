% !TEX TS-program = pdflatex
% !TEX encoding = UTF-8 Unicode

% This is a simple template for a LaTeX document using the "article" class.
% See "book", "report", "letter" for other types of document.

\documentclass[11pt]{article} % use larger type; default would be 10pt

\usepackage[utf8]{inputenc} % set input encoding (not needed with XeLaTeX)

%%% Examples of Article customizations
% These packages are optional, depending whether you want the features they provide.
% See the LaTeX Companion or other references for full information.

%%% PAGE DIMENSIONS
\usepackage{geometry} % to change the page dimensions
\geometry{a4paper} % or letterpaper (US) or a5paper or....
% \geometry{margin=2in} % for example, change the margins to 2 inches all round
% \geometry{landscape} % set up the page for landscape
%   read geometry.pdf for detailed page layout information

\usepackage{graphicx} % support the \includegraphics command and options

% \usepackage[parfill]{parskip} % Activate to begin paragraphs with an empty line rather than an indent

%%% PACKAGES
\usepackage{booktabs} % for much better looking tables
\usepackage{array} % for better arrays (eg matrices) in maths
\usepackage{paralist} % very flexible & customisable lists (eg. enumerate/itemize, etc.)
\usepackage{verbatim} % adds environment for commenting out blocks of text & for better verbatim
\usepackage{subfig} % make it possible to include more than one captioned figure/table in a single float
\usepackage{appendix}
% These packages are all incorporated in the memoir class to one degree or another...

%%% HEADERS & FOOTERS
\usepackage{fancyhdr} % This should be set AFTER setting up the page geometry
\pagestyle{fancy} % options: empty , plain , fancy
\renewcommand{\headrulewidth}{0pt} % customise the layout...
\lhead{}\chead{}\rhead{}
\lfoot{}\cfoot{\thepage}\rfoot{}

%%% SECTION TITLE APPEARANCE
\usepackage{sectsty}
\allsectionsfont{\sffamily\mdseries\upshape} % (See the fntguide.pdf for font help)
% (This matches ConTeXt defaults)

%%% ToC (table of contents) APPEARANCE
\usepackage[nottoc,notlof,notlot]{tocbibind} % Put the bibliography in the ToC
\usepackage[titles,subfigure]{tocloft} % Alter the style of the Table of Contents
\renewcommand{\cftsecfont}{\rmfamily\mdseries\upshape}
\renewcommand{\cftsecpagefont}{\rmfamily\mdseries\upshape} % No bold!



%%% END Article customizations

%%% The "real" document content comes below...

\title{Setup Honeypot: Team Contagiarium}
\author{Mark Vijfvinkel, Aram Verstegen, Joost Kremers \& Bart Lutgens}
%\date{} % Activate to display a given date or no date (if empty),
         % otherwise the current date is printed 

\begin{document}
\maketitle

\section{Introduction}

In this document we explain how we have setup the utility machine and the honeypot. 
We have decided to split up our monitoring time into two parts. In the first part we configure the honeypot as a trapped based system and monitor it for 1 month.
Then we will reconfigure the honeypot to a completely open system and moniter that for 1 month. 
After that period we will analyze the results of both setups and compare the differences in the results we have captured. 
The advantage of a trap based system is that the honeypot can only be compromised to a certain extent and not really used for any malicious activity.
The disadvantage is that it is not possible to capture all the malicious activity an attacker can perform.
The advantage of a completely open system is that you can capture all the malicious activity, but that the system might also be completely compromised and do damage to other systems.


\section{Setup: Utility Machine}

In this section we decribe the setup of the utility machine.

\subsection{Syslog}
Syslog uses UDP to transmit its logfiles to a remote server and this is not very desirable, because everything is sent in plaintext and delivery is not garanteed.
A VPN session between the two hosts is possible, but might be considered as overkill and still does not garantee delivery.
Therefore we have configured syslog to use TCP, which makes it possible to use SSH to tunnel the TCP traffic.
We are using a reverse tunnel from the utility machine to the honeypot.
We use a script to allow the connection to be re-esthablished quickly when the SSH session is terminated for some reason. 
\cite{Remote_logging}

\subsection{Root}

\section{Setup: Honeypot}

In this section we decribe the setup of the honeypot.

\subsection{scanlogd}

To detect portscans we have installed the tool scanlogd.
Per scan an entry is written to syslog.

\subsection{iptables}
To log all connection attempts coming in and going out, we have added the following rules to the iptables configuration. 
This will also log to syslog.

\emph{iptables -I INPUT -m state --state NEW -j LOG --log-prefix "New Connection: "}

\emph{iptables -I OUTPUT -m state --state NEW -j LOG --log-prefix "New Connection: "}


\subsection{Kippo}
To capture attackers that try to come in over ssh, we are using Kippo. 
Kippo is used to log bruteforce attacks and the shell interaction performed by the attacker. 
To make this as realistic as possible we have added a redicrect in the iptables configuration from port 22 to port 2222 that kippo uses. 
Unfortunately this broke down after a while, for reasons yet unknown. 
The reason for forwarding was not to give Kippo root capabilities.

To be able to still log on to the honeypot, we configured port 9022 to be used by ssh.

\subsection{Google Hack Honeypot}
Google Hack Honeypot (GHH) is used to attract search engine attackers. 
These attackers will use search queries to find sensitive data, for example password files, that due to some misconfiguration ended up being indexed by a search engine like google.
To set this up we have first installed Apache and PHP5.
Then we configured GHH to use Comma Seperated Values, since MySQL was not installed yet on the util machine and to see if this would actually work. 
We can later reconfigure to use MySQL, if the results are not satisfactory.
Since GHH offers multiple setup honeypot we picked one to start with, namely GGH version 1.2 passlist.txt.
We had no particular reason for this, only that it was easy to set up. 
The plan was to install more variants of the GHH to attract more attackers, but before this could happen the VM's were shut down.


\subsection{Snort}
Snort is a network intrusion detection system (NIDS) that uses realtime traffic analysis and packet logging on internet protocol networks.
Snort can be used as a network intrusion prevention system, however we will only be using it to detect attacks.
Snort is rule-based and we intend to setup Snort using a predefined rules-set that will only alert on suspicious network behaviour and do nothing to prevent it. 
The rules that we will be using are based on a honeypot rule-set. Included are rules on detecting backdoors, viruses, DDOS, spam, SMTP- and other exploits. There rules will be tweaked if the results are poor.
If a rule is triggered, we can look at the logs from the packet logger to analyse the network traffic.
Snort sends alerts and packet logging files to syslog and as stated earlier, the syslog log files are transfered to the remote util machine.



\bibliographystyle{alpha}
\bibliography{ref_bib}

\end{document}
