\documentclass[11pt]{article}
\usepackage{a4wide, graphicx, fancyhdr, wrapfig, tabularx, amsmath, amssymb, hyperref, color, verbatim, nameref}
\usepackage[english]{babel}
\definecolor{linkcolour}{rgb}{0,0.2,0.6}
\hypersetup{colorlinks,breaklinks,urlcolor=linkcolour, linkcolor=linkcolour}

%----------------------- Macros and Definitions --------------------------

\setlength\headheight{20pt}\usepackage{}
\addtolength\topmargin{-10pt}
%\addtolength\footskip{20pt}

\fancypagestyle{plain}{%
\fancyhf{}
\fancyfoot[RO,LE]{\sffamily\bfseries\thepage}
\renewcommand{\headrulewidth}{0pt}
\renewcommand{\footrulewidth}{0pt}
}

\pagestyle{fancy}
\fancyhf{}
\fancyfoot[RO,LE]{\sffamily\bfseries\thepage}
\fancyhead[RO,LE]{\textsc{}}
\fancyhead[LO,RE]{\emph{}}
\renewcommand{\headrulewidth}{1pt}
\renewcommand{\footrulewidth}{0pt}
\newcommand{\tab}{\hspace*{2em}}

\newcommand{\tocheck}[1]{{\bf !?: #1 :!?}}
%\newcommand{\OBA}{Online behavioural advertising }

\frenchspacing

%-------------------------------- Title ----------------------------------

\title{\textbf{Honeypot Assignment: \\ \emph{Group Contagiarium}}}
\author{Aram Verstegen, \tocheck{studentnummer Utwente} \\
	 Mark Vijfvinkel, 1275194 \\
	 Bart Lutgens, 12752408 \\
	Joost Kremers, 1275259}

\date{\today}

\begin{document}
\maketitle


\section{Introduction}
\tocheck{Kijken of the secties nog kloppen!}
In this paper we will try to present an overview of the data we have collected from the honeypot and explain the different attacks that were waged against it. 
In the second section we will describe the changes we have made to the setup during the "Waiting for hackers" phase. 
The third section will present an overview of various statistics we have collected from the logfiles, e.g. number of IP-addresses, countries and attack types. 
In the fourth section we will discuss which attacks took place against the different services the honeypot was running e.g. SSH and MySQL.
In order to discover the sophisticationof the attacks, we will try to present possible connections between an attack on one service and an attack on a different service, in the fifth section.
The sixth section will conclude our findings. 
Finally we will discuss this project and provide some improvements. 




\section{Setup}
\label{Setup}

On the 1st of November we delivered our initial setup document, however during the "Waiting for hackers" phase we have made adjustments to the setup in order to attract more activity to the honeypot. In this section we will present the initial setup and in addittion to that, discuss the changes that were made.

%%%%MOET NOG WORDEN HERZIEN%%%%%

\subsection{Monitoring Machine}

In this subsection we describe the setup of the monitoring machine.

\subsubsection{Syslog-ng}
By default, \verb|syslog| uses UDP to transmit its log messages to a remote server.
Relying on UDP is not desirable, because everything is sent in plaintext and delivery of UDP messages is not guaranteed.
To allow for confidential and reliable transmission of the syslog messages from the one machine to another, one needs to use TCP connections in syslog, and provide some way of encapsulating the socket used in a cryptographic layer.


We decided to set up a reverse SSH tunnel from the monitoring machine to the honeypot machine, as described in the article "Remote logging with SSH and syslog-ng''. \cite{remote_logging}
The example command establishes an SSH connection the honeypot, setting options to forego a login shell or anything a normal user would desire, and binds the local socket 9514 on the honeypot machine to tunnel its connections to the same port on our monitoring machine.

We configured a locally installed \verb|syslog-ng| package for our user on the monitoring machine and have configured it to listen to local TCP connections on port 9514 and will thereby receive syslog messages transmitted through this port.
%We are aware there is still the possibility of other users on the monitoring machine poisoning our logs, but we feel we can rule out foul play by our peers in this assignment.

To allow for secure SSH access, a public keypair was generated on the util machine and the associated public key was added to the honeypot's \verb|authorized_keys| file.

We have put the command to create the SSH tunnel in a script to allow the connection to be re-esthablished quickly and easily if the SSH session is terminated for some reason. 
A severance of the connection would lead us to assume the worst initially.
If re-establishing this connection fails after any attackers have gained access (which should never happen if we monitor adequately), we would have assumed that the machine was compromised beyond our control and would have requested a shutdown of the VM, if no network downtime was otherwise announced.

\subsubsection{Monitoring}
After collecting our syslog information, we wanted to be somehow alerted of interesting information automatically.
We decided to use an IRC bot that would leave a message in an IRC channel (\#contagiarium) everytime an entry was made to the syslog file. First this lead to an overloaded channel and several false positives. To solve these problems filters (regular expressions in a file) were created to only post relevant information. For example, from the kippo logs only the actual login attempts would be shown with the tried password.


\subsection{Honeypot}

In this section we describe the setup of the honeypot.

\subsubsection{Rsyslog}
On the honeypot, we modified the default \verb|rsyslog| configuration to also send syslog messages to the (now bound) local port 9514, completing the chain to the syslog instance on our monitoring machine.
At first the modified configuration was stored in its own file in the syslog.d directory, which is a bit obvious to anyone taking an interest. Therefore we moved the configuration to the standard syslog directory. 

%We plan to move this small configuration over to one of the default configuration files to avoid raising suspicion outright.
Note that the chosen port for remote syslog is very similar to the default syslog port 514, which might have given us away. So we changed the port number to 9514.
%We plan to change it to a more nondescript port number.

\subsubsection{SSHFS mount}
Not all services support logging to syslog and for some it is infeasible.
We created a read-only \verb|sshfs| mount from our monitoring machine that we used to write our logfiles for various services to.
We used an interactive login to mount this, as allowing the honeypot back into our monitoring machine with an SSH keypair just seems dangerous.

\subsubsection{scanlogd}
To detect port scans we installed the service \verb|scanlogd|.
Per detected scan an entry is written to syslog.

\subsubsection{iptables}
To log all incoming and outgoing connection attempts, we have added the following rules to the iptables configuration. The attempts were also logged to syslog.

\begin{verbatim}
iptables -I INPUT -m state --state NEW -j LOG --log-prefix "New Connection: "
iptables -I OUTPUT -m state --state NEW -j LOG --log-prefix "New Connection: "
\end{verbatim}

We have filtered outgoing SMTP, SIP and NetBIOS traffic, in order to prevent abuse and spam.
%We are also going to filter outgoing SMTP, SIP and NetBIOS traffic.

\subsubsection{Kippo}
To capture attackers that try to come in by brute-forcing SSH logins, we used Kippo. 
Kippo is a sandbox SSH service, used to log bruteforce attacks and the shell interaction performed by the attacker after gaining access (with a trivial password). 
The latest development version was installed to run as a separate user.

To make this as realistic as possible we added a redirect in the iptables configuration from the real SSH port 22 to port 2222 that Kippo uses by default.
The reason for forwarding in iptables was to not give Kippo root rights, which would be very dangerous.

%Unfortunately the forward broke down around the time the firewall was being reconfigured, for reasons yet unclear . 
%If it can't be resolved, we can circumvent this issue by running Kippo through a copy of the Python interpreter that has the capability to bind privileged ports.

To be able to still log on to the honeypot, we configured our real SSH service to run on port 9022.

Furthermore, we patched Kippo to log messages to syslog.

During the second phase of the project, we noticed that of the succesfull bruteforce attempts against Kippo, none actually tried to continue and explore or use the honeypot. Running a default portscan with Metasploit revealed that it can detect if Kippo is being used. We thought this explained why nobody tried to log in. To solve this problem, we adjusted the Kippo configuration so that Metasploit could not recognize Kippo. Unfortunately this did still not lead to actual exploration or use of the honeypot. 

\subsubsection{DenyHosts}
If someone found out that our real SSH port was changed to 9022, they might try to break in through the real SSH service.
To make sure no bruteforce attacks could be done on this port, DenyHosts was installed.
This service blocks IP-addresses that try to log in more than 3 times with the wrong password by adding them to the \verb|hosts.deny| file.
The hosts listed there are blacklisted on the machine, but we can edit them if needed.

%Denyhosts can log to syslog by editing its configuration file.

\subsection{Apache \& Nginx}
To attract web traffic, we installed the Apache web server, the Apache PHP runtime and the \verb|mod-security| module which allows us to spoof its version information, to make it appear old and vulnerable.
The Nginx webserver is installed as a reverse proxy in front of Apache, which runs on local port 8080.
This provides 2 layers of logging (to files), and Nginx's superior transparent caching mechanism allows us to avoid exhausting our resources with scripting languages, should we have ever faced an application-layer DoS attack (Nginx appears quite resilient against those).

Nginx has been compiled from source and was patched to send the spoofed Apache version header as well.
We will configure both services to log to our \verb|sshfs| mount.

\subsubsection{Google Hack Honeypot}
Google Hack Honeypot (GHH) is designed to be used to attract search engine attackers. 
These attackers will use search queries to find sensitive data, for example password files, that due to some misconfiguration ended up being indexed by a search engine like Google.
GHH allows logging of these requests to a MySQL database or Comma Seperated Value (CSV) file.

Since GHH offers multiple search targets, we picked one to start with, namely GGH version 1.2 passlist.txt.
We had no particular reason for this, only that it was easy to set up. 
%We plan to install more variants of the GHH to attract more attackers, but before this could happen the VM's were shut down.

%To set up GHH we have first installed Apache and PHP5.
We configured GHH to log to in CSV format, since it requires minimal setup and is easily parsed.
%We can later reconfigure to use MySQL, if the results are not satisfactory.
Then, we patched the GHH code to allow logging to syslog, to maintain a single, consistent point of log collection.

We hoped to detect and capture a bruteforce attack or scan on generic filenames, for example \verb|passwd.txt|, because this would probably have warned us that an attacker will try to use a dictionary attack.
Unfortunately GGH broke down after the Virtual Machines were shut down, we were unable to get it back up running and therefore decided not invest more time and effort into it. We also believe it would not have brought us more interesting data.

\subsubsection{Website}
To attract more activity we decided to create a fake website that would resemble a website of the University of Twente. To make it more appealing in ways of computing power we used the department of Partical Physics and Astronomy as a cover.
On the website we also put some information about a fictive person (Jenny Abott) would manage the servers and website. So we created a useraccount with the same login and password, namely Jenny. This way a more sophisticated attack would be possible. An attacker would look at the website, determine if it is intresing to attack and then connect to Kippo to log in as Jenny.

\subsubsection{Wordpress}
In order to make the website more vunerable and appealing to hackers, we installed an out of date version of Wordpress. Wordpress is software, powered by PHP and MySQL, that can be used to easily create websites or blogs and manage them via the content management system. Exploits are found on a frequent basis for Wordpress and therefore patches are published frequently. In the first week of December we installed version 2.8.4 of Wordpress on the honeypot.

\subsubsection{Snort and Suricata}
Snort is an open source intrusion detection system (IDS) that uses realtime traffic analysis and packet logging on internet protocol networks.
Snort was designed to be used as a network IDS, however we will be running it on our host machine, effectively making it a host-based IDS.

We set up Snor using the community rule-set that will only alert on suspicious network behaviour, but do nothing to prevent it. 
Included are rules on detecting exploit attempts on various common services like HTTP, FTP, SMTP, and many more.

We configured Snort to send alerts to syslog, and to log its packet captures to the read-only \verb|sshfs| mounted from our monitoring machine.
We also installed \verb|oinkmaster| to automatically retrieve the latest community Snort rules, with a Snort API key registered specifically for this project.

Unfortunately we noticed that Snort required a lot of resources and was slowing down the honeypot. Instead of Snort we decided to use Suricata, which uses the same ruleset, but uses it more efficiently and therefore less resources.

\subsubsection{Dionaea}
Dionaea is a low interaction honeypot that provides several services, namely http, ftp, tftp, MSSQL, MySQL and SIP (VoIP). We installed this to enable more services and attract more traffic to the honeypot. 

%Considering we are in a VM we might be limited in RAM, which Snort seems to require a lot of.
%The configured rules will have to be minimal to avoid exhausting resources.

\subsubsection{Auditing}
During the High Interaction phase we used a kernel module that hooked execve (new process) calls and commandline interaction. This information was logged to syslog which provided us with auditing information during this phase.
%For the HI phase, we would like to have a way to audit commandline interaction. 
%Various options exist, such as the user-space \verb|auditd| daemon, but to make auditing harder to detect this we feel it would be best to look for a kernel-space solution.
%An auditing module developed by the Honeynet project called Sebek can provide this, but is unmaintained and outdated.
%The grsec kernel patches also provide auditing from kernel space, but we were advised against replacing the stock kernel.

%We are still investigating the best way to achieve useful auditing.
%A self-written kernel module that hooks system calls, albeit limited, might suffice to get some much needed auditing information from the honeypot.


\section{Statistics}
\label{Statistics}
In this section we will discuss the statistical data that have been gathered from our logfiles. 



\section{Attacks on the different services}
\label{Attacks}
During phase 2 we have monitored the honeypot closely and collected data about different attacks against the honeypot. In this section we will discuss the attacks against the services the honeypot was running.

\begin{itemize}
\item Kippo

The first service we will be talking about is Kippo. During the interaction phase we've had 2610 login attempt on our Kippo client and 3871 connections. 54 of these attempts were succesfull.
We have made a top 5 of most tried usernames in these login attempts:

\begin{itemize}
\item   2206 root
\item     39 admin
\item     26 bin
\item     20 w000t
\item     19 oracle
\item     19 jenny
\end{itemize}

It is not entirely unexpected that the username "root" is tried most ofted. 
We assume the username "w000t" is a backdoor user created in one or more exploits.
Interesting to note is that "jenny" is the name of a fake employee listed on the our honeypot website.
Most usernames were tried multiple times, even if they made no sense to us (for instance msnayeem_iitd tried to log in four times).

These are the top 5 passwords tried:

\begin{itemize}
\item     55 
\item     28 123456
\item     20 111111
\item     16 root
\item     16 admin
\end{itemize}

As you can see, the most tried password is a blank password. The other passwords in this top 5 are all standard password that can be found in a dictionary attack. It is nice to note that 123456 was accepted by our system.

An interesting situation was someone that tried to log in to the honeypot with the user root, just trying a list of years (1960-1980). He obviously thought we would use a birthyear as our password (which was not the case).

\item HTTP
In three weeks, a total of 76002 HTTP-requests were received by our Apache server originating from 176 unique IP addresses. 4369 of these requests were 404'd.

<<<<<<< HEAD
These 404's are interesting because they show us what adresses the attackers were trying to access. For instance, we see a lot of scans for phpMyAdmin (phpMyAdmin-3.3.10.5, phpMyAdmin-3.4.7, phpMyAdmin-3.4.6 etcetera). They also searched non existing folders for these phpMyAdmin files. Also serveral client IDs are visible in these logs: "\verbatim{"-"}", which is a blank user/client ID, (3432 attempts), "MiRRORS" (310 attempts), "ZmEu" (41 attempts) and "Morfeus Fucking Scanner" (4 attempts). These seem to be search engine crawlers. It is interesting to see that both several "\verbatim{"-"}" and several "ZmEu" users are scanning for "/w00tw00t.at.blackhats.romanian.anti-sec:)" and "/w00tw00t.at.ISC.SANS.DFind:)". The attackers using "ZmEu" found one valid path: /phpmyadmin/, however he did not exploit this. Morfeus Fucking Scanner was used by one attacker, that soly searched for /user/soapCaller.bs. We could not find anything on this .bs file. We expected to see several occurances of "/robots.txt" because this is a common way to start when using a search engine crawler, but none were found.
=======
These 404's are interesting because they show us what adresses the attackers were trying to access. For instance, we see a lot of scans for phpMyAdmin (phpMyAdmin-3.3.10.5, phpMyAdmin-3.4.7, phpMyAdmin-3.4.6 etcetera). They also searched non existing folders for these phpMyAdmin files. 

Also serveral client IDs are visible in these logs: "\verbatim{"-"}", which is a blank user/client ID, (3432 attempts), "MiRRORS" (310 attempts), "ZmEu" (41 attempts) and "Morfeus Fucking Scanner" (4 attempts). These seem to be search engine crawlers. It is interesting to see that both several "\verbatim{"-"}" and several "ZmEu" users are scanning for "/w00tw00t.at.blackhats.romanian.anti-sec:)" and "/w00tw00t.at.ISC.SANS.DFind:)". The attackers using "ZmEu" found one valid path: /phpmyadmin/, however he did not exploit this. "Morfeus Fucking Scanner" was used by one attacker, that soly searched for /user/soapCaller.bs. We could not find anything on this .bs file. We expected to see several occurances of "/robots.txt" because this is a common way to start when using a search engine crawler, but none were found.

>>>>>>> b23ff5541ec52f98e7cb8761679e0580b63fe136


\section{Connection between the attacks}
\label{Connection}
In this section we will correlate the data about attacks against one service against the data of attacks of a different service. We will try to look for a pattern or similarities in this data and determine if the attack was more sophisticated.


\section{Conclusion}
\label{Conclusion}
From Section \ref{Connection} we can conclude that our honeypot was not as popular as we had hoped it would be. Because we chose not to advertise, we solely relied on the hackers abilities to find us by for example portscans. The number of attacks on us was therefore not high and neither was the depth of the attacks. 

Our honeypot was not as popular as we had hoped it would be. Because we chose not to advertise, we solely relied on the hackers abilities to find us by for example portscans. The number of attacks on us was therefore not high and neither was the depth of the attacks. 

Concluding: 


\section{Discussion}
\label{Discussion}
In this section we would like to discuss some issues we encountered during the honeypotproject. 

\begin{itemize}
\item Low/High interaction phase

To see what the best approach is to a honeypot project, we decided to split the project in a Low and a High interaction phase. Unfortunately the High interaction phase yielded not much more results than the low interaction phase. Additionally we were a little late in switching to the HI phase. So in hindsight we don't think it was worth switching to the high interaction phase.

\item Lack of interest

We decided to make this honeypot as realistic as possible, this meant: not advertising our honeypot on IRC-channels or 4Chan. Our motivation was that in nonhoneypot circumstances this would also not be done. So for example advertising the IP address of the honeypot would attract a lot more hackers than a normal server would. So the results would not be on par with attacks against a normal server. In the High interaction phase we had a website that attracted people to our honeypot, which is a more realistic approach to attracting hackers.

\item DenyHosts

To make sure our real SSH port would not be brute forced, we added DenyHost to the port. We soon realized that this was a very effective countermeasurement versus intruders: both Bart and Joost were blocked due to their lack of Linux knowledge. Luckely Mark and Aram were able to unblock the other two.

\item Linux knowledge

Two of the teammembers had limited Linux experience. It took some time for them to understand how Linux commandline works. Fortunately we had two guru's in the group that were able to teach the newbies.

\item Google Hack Honeypot

The Google Hack Honeypot was supposed to attract hackers. Unfortunately it soon crashed due to unknown configuration errors. Because the results in the period the tool was running were nonexistent and we were switching to HI later on, we decided not to use it. 

<<<<<<< HEAD
\end{itemize}

\begin{itemize}
\item Low/High interaction phase
\item GHH
\item etc
=======
>>>>>>> b23ff5541ec52f98e7cb8761679e0580b63fe136
\end{itemize}



%%% REFS %%%

\bibliographystyle{plain} % amsalpha
\bibliography{ref_honeypot}

\end{document}
